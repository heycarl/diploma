\section{Системное проектирование}

В рамках данного дипломного проекта разрабатывается отказоустойчивая корпоративная \textit{IT}-инфраструктура, которая включает несколько ключевых модулей:
\begin{itemize}
    \item модуль управления пользователями;
    \item модуль мониторинга;
    \item модуль хранилища секретов;
    \item модуль \textit{Kubernetes};
    \item модуль работы с облачной инфраструктурой;
    \item модуль \textit{VPN};
    \item модуль автоматизации \textit{GitOps};
    \item модуль управления конфигурацией.
\end{itemize}

Модули обеспечивают высокую доступность, безопасность и удобство для сотрудников компании. Каждый модуль решает конкретную бизнес-задачу, обеспечивая эффективное функционирование всей инфраструктуры. Структурная схема  отказоустойчивой корпоративной \textit{IT}-инфраструктуры компании виде диаграммы компонентов на чертеже ГУИР.400201.030 C1. 

\subsection{Модуль управления пользователями} 
\label{sec:user_management_module}

Модуль управления пользователями решает задачу централизованного контроля за доступом сотрудников к корпоративным ресурсам. Он реализует систему \textit{IDM}, которая управляет учетными записями и правами доступа пользователей. Задача этого модуля -- обеспечить безопасность и соответствие политике доступа, гарантируя, что только авторизованные пользователи могут получить доступ к критически важным данным и сервисам.

Высокоуровневая бизнес-задача, которую решает этот модуль, заключается в обеспечении защиты корпоративных ресурсов и данных путем надежного и эффективного управления доступом, что напрямую влияет на безопасность и соблюдение стандартов компании.

\subsection{Модуль мониторинга} 
\label{sec:monitoring_alerting_module}

Модуль мониторинга и оповещений решает задачу непрерывного наблюдения за состоянием всех ключевых компонентов инфраструктуры. Он собирает метрики и логи, анализирует их, а также настраивает систему уведомлений и оповещений в случае возникновения сбоев или нарушений работы сервисов. Задача этого модуля -- гарантировать высокую доступность и минимизировать простои сервисов, своевременно информируя команду о любых отклонениях, что напрямую влияет на производительность и бесперебойную работу бизнеса.

Высокоуровневая бизнес-задача, которую решает этот модуль, заключается в обеспечении непрерывной и бесперебойной работы инфраструктуры путем своевременного выявления и устранения потенциальных проблем, что критично для обеспечения максимальной производительности и надежности сервисов компании.

Модуль мониторига имеет логическую связность:
\begin{itemize}
    \item с модулем управления пользователями для получения информации о пользователях, их правах и контактных данных для оповещений;
    \item c модулем \textit{Kubernetes}, используя его как одну из целей для мониторинга;
    \item с модулем \textit{VPN} для сбора информации о подключениях пользователей.
\end{itemize}

\subsection{Модуль хранилища секретов} 
\label{sec:secrets_storage_module}

Модуль хранилища секретов решает задачу безопасного хранения и управления конфиденциальной информацией, такой как ключи, сертификаты и пароли. Используя систему \textit{Vault}~\cite{hashicorpvault}, этот модуль обеспечивает централизованный доступ к данным и гарантирует их защиту. Важной бизнес-задачей является минимизация рисков утечек информации, что критично для поддержания безопасности корпоративной \textit{IT}-инфраструктуры и соблюдения норм по защите данных.

Высокоуровневая бизнес-задача, которую решает этот модуль, заключается в обеспечении защиты конфиденциальных данных компании, снижении рисков утечек и утрат, что непосредственно влияет на безопасность всей инфраструктуры.

Модуль хранилища секретов имеет логическую связность с модулем управления пользователями для получения информации о пользователях и их логических правах.

\subsection{Модуль \textit{Kubernetes}} 
\label{sec:kubernetes_cluster}

\textit{Kubernetes} кластер решает задачу оркестрации контейнеризированных приложений, обеспечивая автоматизацию их развертывания, масштабирования и управления. Этот модуль играет ключевую роль в обеспечении отказоустойчивости и высокодоступности приложений. Задача \textit{Kubernetes} -- обеспечить масштабируемость и быстрое восстановление сервисов, что способствует росту бизнеса и позволяет компании оперативно адаптироваться к изменениям нагрузки и бизнес-требованиям.

Высокоуровневая бизнес-задача, которую решает этот модуль, заключается в обеспечении высокой доступности и гибкости инфраструктуры путем эффективной оркестрации контейнеризированных приложений, что напрямую влияет на способность компании быстро реагировать на изменения в бизнес-условиях.

Модуль \textit{Kubernetes} имеет логическую связность с модулем управления конфигурацией, что бы влиять на процесс эксплуатации кластера.

\subsection{Модуль работы с облачной инфраструктурой} 
\label{sec:cloud_infrastructure_module}

Модуль работы с облачной инфраструктурой решает задачу управления виртуальными ресурсами с помощью принципов \textit{IaC (Infrastructure as Code)}. Он позволяет создавать, настраивать и управлять облачными сервисами и виртуальными машинами. Основной бизнес-задачей этого модуля является ускорение процесса внедрения и настройки новых ресурсов, что помогает компании масштабировать свою инфраструктуру по мере роста бизнеса и быстро реагировать на изменения в требованиях.

Высокоуровневая бизнес-задача, которую решает этот модуль, заключается в обеспечении эффективного и быстрого развертывания инфраструктуры, что позволяет компании быть гибкой и оперативно адаптироваться к изменениям в бизнес-среде.

\subsection{Модуль \textit{VPN}} 
\label{sec:vpn_access_module}

Модуль управления доступом через \textit{VPN} решает задачу обеспечения защищенного удаленного доступа сотрудников к корпоративной сети. Этот модуль шифрует трафик и гарантирует конфиденциальность передаваемых данных. Его задача -- обеспечить безопасность при удаленной работе, что критично для гибкости компании и сохранения высокой производительности сотрудников в условиях удаленной работы или командировок.

Высокоуровневая бизнес-задача, которую решает этот модуль, заключается в обеспечении безопасности и мобильности сотрудников путем предоставления безопасного доступа к корпоративной сети, что влияет на эффективность работы компании в условиях гибридной или удаленной занятости.

\subsection{Модуль автоматизации \textit{GitOps}} 
\label{sec:gitops_automation_module}

Модуль автоматизации с помощью \textit{GitOps} решает задачу упрощения процессов развертывания и обновления приложений. Он интегрирует процесс разработки с процессом развертывания, позволяя разработчикам и операционным командам управлять инфраструктурой через 
\textit{Git} репозитории. Задача этого модуля -- ускорить доставку новых версий приложений и минимизировать человеческие ошибки при развертывании, что способствует повышению качества и надежности сервисов.

Высокоуровневая бизнес-задача, которую решает этот модуль, заключается в ускорении и оптимизации процессов развертывания приложений, что непосредственно влияет на скорость и качество предоставляемых услуг и продуктов компании.

Модуль автоматизации \textit{GitOps} имеет логическую связность с модулем \textit{Kubernetes} для установки и настройки компонентов \textit{IT}-инфраструктуры, а так же связность с модулем мониторинга.

\subsection{Модуль управления конфигурацией} 
\label{sec:configuration_management_module}

Модуль управления конфигурацией отвечает за автоматизацию настройки серверов, сервисов и программного обеспечения с использованием инструментов управления конфигурациями, таких как \textit{Ansible}, а также систем контроля версий, таких как \textit{Git}. Он позволяет централизованно и воспроизводимо управлять состоянием инфраструктуры, исключая необходимость ручного вмешательства при выполнении повторяющихся задач.

Основной бизнес-задачей данного модуля является повышение надёжности и предсказуемости при внедрении изменений, что особенно важно при масштабировании инфраструктуры или обслуживании нескольких окружений. Использование подхода \textit{Idempotent Configuration Management} обеспечивает согласованное состояние системы вне зависимости от текущего состояния узлов.

Высокоуровневая цель, которую решает модуль -- это обеспечение стабильного и контролируемого управления конфигурациями в масштабируемой инфраструктуре. Это способствует снижению человеческого фактора, ускорению процессов обновления и улучшению безопасности за счёт прозрачности и отслеживаемости всех изменений.

Модуль конфигурацией имеет логическую связность с модулем управления облачной инфраструктрой, что бы использовать и конфигурировать созданные виртуальные объекты инфраструктуры.
