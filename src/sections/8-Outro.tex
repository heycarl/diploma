\sectionCenteredToc{Заключение}
\label{sec:outro}

В рамках выполнения настоящего дипломного проекта была разработана и реализована программная инфраструктура, предназначенная для автоматизированного развертывания и сопровождения облачного окружения на базе \textit{Kubernetes}. Решение охватывает полный цикл жизнедеятельности кластерного программного обеспечения: от первичной инициализации серверов до конфигурации мониторинга и \textit{CI/CD}-процессов. В качестве ключевых компонентов системы использованы: инструмент автоматизации \textit{Ansible}, система контроля версий \textit{GitLab} с интеграцией \textit{GitOps}-подхода, а также стек мониторинга.

Разработанное решение обладает следующими преимуществами:
\begin{itemize}
    \item модульная архитектура, реализованная в виде набора \textit{Ansible}-ролей, обеспечивающих повторяемость и масштабируемость развертывания;
    \item использование \textit{GitLab CI} для контроля выполнения плейбуков и обеспечения воспроизводимости всех стадий конфигурации;
    \item полная автоматизация установки и настройки \textit{Kubernetes}-кластера с разделением на роли \textit{bootstrap}, \textit{base-node}, \textit{k3s};
    \item внедрение мониторинга и визуализации с помощью \textit{Helm}-чарта \textit{victoria-metrics-k8s-stack};
    \item централизованное хранение логов и метрик с возможностью масштабирования;
    \item соблюдение принципов инфраструктуры как кода, обеспечивающих прозрачность и контролируемость изменений.
\end{itemize}

Поставленные задачи проекта выполнены в полном объеме. Разработанная система успешно автоматизирует ключевые процессы сопровождения \textit{Kubernetes}-кластера, повышая надежность и ускоряя развертывание окружений. В качестве возможных направлений развития проекта можно выделить:
\begin{itemize}
    \item интеграцию дополнительных средств защиты и авторизации, в том числе через \textit{Policy-as-Code};
    \item добавление механизмов самовосстановления и автоматической реакции на аномалии;
    \item внедрение более гибкой системы параметризации и шаблонизации окружений;
    \item расширение возможностей \textit{CI/CD} за счет оркестрации \textit{multi-cluster} деплоев;
    \item построение распределенного мониторинга с агрегацией метрик и логов на уровне нескольких дата-центров.
\end{itemize}

Таким образом, реализованное решение демонстрирует практическую применимость современных подходов к управлению инфраструктурой, а также обладает потенциалом к масштабированию, адаптации под различные сценарии эксплуатации и интеграции с внешними сервисами.
