\section{Обзор литературы}

\subsection{\textit{GitOps}}
\label{sec:gitops}
\textit{GitOps} — это подход к автоматизации и управлению инфраструктурой, в котором используется система контроля версий \textit{Git} как единственный источник истины для всей инфраструктуры и приложений. Этот метод активно используется для управления \textit{Kubernetes}-кластерами и другими контейнеризованными приложениями. В отличие от традиционных методов управления инфраструктурой, где операции выполняются вручную или с помощью инструментов, таких как \textit{Ansible} или \textit{Terraform}, в \textit{GitOps} все изменения конфигурации управляются через \textit{Git}. 

\subsubsection{История и развитие \textit{GitOps}.}
Подход GitOps был предложен в 2017 году компанией \textit{Weaveworks} и быстро получил признание среди разработчиков, работающих с \textit{Kubernetes}. Идея заключалась в том, чтобы использовать один инструмент — \textit{Git} — для всего цикла жизни инфраструктуры. Это позволило упростить процессы управления конфигурациями и повысить безопасность, поскольку все изменения теперь могли быть отслежены и автоматически применены через \textit{CI/CD} пайплайны.

\subsubsection{Применение \textit{GitOps}.}
Этот подход применим в средах, где используется контейнеризация и микросервисная архитектура. Он используется для автоматического применения изменений в инфраструктуре на основе изменений в репозиториях \textit{Git}, что позволяет упростить развертывание новых версий приложений и настройку инфраструктуры.

\subsection{\textit{Git}}
\label{sec:git}
\textit{Git} — это распределенная система контроля версий, изначально разработанная Линусом Торвальдсом в 2005 году для поддержки разработки ядра операционной системы \textit{Linux}. \textit{Git} позволяет разработчикам отслеживать изменения в коде, работать над проектами в команде, и объединять изменения в единое целое.

\subsubsection{Особенности и принципы работы \textit{Git}.}
Одной из основных особенностей \textit{Git} является его распределенная природа. В отличие от централизованных систем контроля версий, таких как \textit{Subversion}, где существует один центральный репозиторий, \textit{Git} позволяет каждому пользователю хранить локальную копию репозитория. Это ускоряет процессы работы с кодом, повышает безопасность и делает систему более устойчивой к сбоям.

\subsubsection{Git в контексте управления инфраструктурой.}
\textit{Git} активно используется в подходе \textit{GitOps}, как основной инструмент для автоматизации развертывания инфраструктуры и приложений. Все изменения, касающиеся настройки инфраструктуры или конфигурации приложений, коммитятся в репозиторий \textit{Git}, откуда автоматически распространяются на рабочие системы, через инструменты \textit{CI/CD}, такие как \textit{Jenkins} или \textit{GitLab CI}.

\subsection{\textit{Kubernetes}}
\label{sec:kubernetes}
\textit{Kubernetes} — это система оркестрации контейнеров с открытым исходным кодом, разработанная компанией \textit{Google} и переданная в \textit{Cloud Native Computing Foundation} (CNCF) в 2014 году. \textit{Kubernetes} предоставляет средства для автоматизации развертывания, масштабирования и управления контейнеризованными приложениями в различных средах — от локальных серверов до облачных инфраструктур.

\subsubsection{История \textit{Kubernetes}.}
Разработка \textit{Kubernetes} началась в \textit{Google} как внутренний проект, на основе системы управления контейнерами \textit{Borg}, которая использовалась для управления вычислительными ресурсами в масштабах всей компании. В 2014 году проект был выпущен как открытый исходный код, и с тех пор он стал де-факто стандартом для оркестрации контейнеров в облачных и локальных инфраструктурах.

\subsubsection{Основные компоненты \textit{Kubernetes}.}
\begin{itemize}
    \item \textit{Pod} — наименьшая и базовая единица развертывания в \textit{Kubernetes}, которая может содержать один или несколько контейнеров.
    \item \textit{ReplicaSet} — гарантирует наличие нужного числа реплик \textit{Pod}-ов, что обеспечивает отказоустойчивость.
    \item \textit{Deployment} — управляет версиями и обновлениями приложения, развернутого в контейнерах.
    \item \textit{Service} — абстракция, которая определяет, как контейнеры могут взаимодействовать друг с другом и с внешними клиентами.
    \item \textit{Ingress} — управляет входящими HTTP и HTTPS запросами в кластер.
\end{itemize}

\subsubsection{Применение \textit{Kubernetes}.}
\textit{Kubernetes} широко используется для автоматизации развертывания и масштабирования микросервисных приложений. Он позволяет эффективно управлять большими кластерами контейнеров и обеспечивает высокую доступность и отказоустойчивость приложений. \textit{Kubernetes} является неотъемлемой частью большинства современных облачных инфраструктур и используется во многих крупных организациях.

\subsection{\textit{Terraform}}
\label{sec:terraform}
\textit{Terraform} — это инструмент для управления инфраструктурой как кодом (\textit{IaC}), с открытым исходным кодом, который позволяет пользователям описывать инфраструктуру с помощью декларативных конфигурационных файлов. Он был разработан компанией \textit{HashiCorp} и позволяет управлять ресурсами облачных провайдеров, такими как \textit{AWS}, \textit{Google Cloud}, \textit{Azure}, а также локальными ресурсами.

\subsubsection{Особенности и использование \textit{Terraform}.}
Одной из ключевых особенностей \textit{Terraform} является способность управлять инфраструктурой на различных облачных платформах через единый интерфейс. \textit{Terraform} использует язык конфигурации \textit{HCL (HashiCorp Configuration Language)}, который прост в освоении и позволяет описывать инфраструктуру декларативным способом. Инструмент создает, изменяет и отслеживает состояние инфраструктуры, что позволяет автоматизировать процессы развертывания и управления ресурсами.

\subsection{\textit{Ansible}}
\label{sec:ansible}
\textit{Ansible} — это инструмент автоматизации для управления конфигурациями, развертывания приложений и задач по оркестрации. Он был разработан компанией \textit{Red Hat} и используется для автоматизации рутинных операций и ускорения процессов развертывания.

\subsubsection{Особенности и принципы работы \textit{Ansible}.}
Одним из главных преимуществ \textit{Ansible} является его простота в использовании. Он использует простой язык \textit{YAML} для описания конфигураций и задач, что делает его доступным для администраторов и разработчиков. \textit{Ansible} не требует агентов на целевых машинах, что упрощает настройку и управление.

\subsubsection{Применение \textit{Ansible}.}
\textit{Ansible} активно используется для автоматизации развертывания приложений, управления конфигурациями серверов, а также оркестрации процессов в инфраструктуре.

\subsection{\textit{IDM (Identity Management)}}
\label{sec:idm}
Система управления идентификацией \textit{(IDM)} предназначена для централизованного управления учетными записями пользователей и их доступом к корпоративным ресурсам. \textit{IDM} позволяет автоматизировать процессы создания, изменения и удаления учетных записей, а также назначать права доступа к различным системам.

\subsubsection{Основные функции \textit{IDM}.}
Системы управления идентификацией включают в себя следующие основные функции:
\begin{itemize}
    \item Регистрация пользователей и управление их аттрибутами.
    \item Аутентификация и авторизация пользователей.
    \item Управление ролями и правами доступа.
    \item Отслеживание событий и журналирование действий пользователей.
\end{itemize}

\subsection{Мониторинг и алертинг (\textit{Grafana Stack})}
\label{sec:monitoring}
Для обеспечения стабильности и высокой доступности инфраструктуры важным аспектом является мониторинг. \textit{Grafana Stack} включает в себя \textit{Prometheus}, который собирает и хранит метрики, и \textit{Grafana}, которая визуализирует данные и отображает их в виде панелей мониторинга.

\subsubsection{Использование \textit{Grafana} для мониторинга.}
\textit{Grafana} позволяет создавать динамические и настраиваемые дашборды для мониторинга состояния инфраструктуры, включая показатели нагрузки на серверы, использование ресурсов, ошибки и другие метрики. \textit{Prometheus} обеспечивает сбор метрик с приложений, серверов и других компонентов инфраструктуры.

\subsection{\textit{VPN (OpenVPN)}}
\label{sec:vpn}
\textit{OpenVPN} — это популярное решение для создания виртуальных частных сетей (\textit{VPN}), которое позволяет пользователям безопасно подключаться к корпоративной сети через интернет.

\subsubsection{Как работает \textit{OpenVPN}.}
\textit{OpenVPN} использует протоколы шифрования для защиты данных, передаваемых через общественные сети. Он поддерживает различные методы аутентификации и шифрования, включая сертификаты и двухфакторную аутентификацию, что повышает безопасность подключения.

\subsubsection{Применение \textit{OpenVPN}.}
\textit{OpenVPN} используется для обеспечения безопасного удаленного доступа к корпоративной сети, что особенно важно для организаций с распределенными командами и удаленными сотрудниками.

\subsection{Обзор существующих аналогов}
\label{sec:existing_analogs}
На современном рынке существует множество облачных платформ и инструментов для автоматизации развертывания инфраструктуры и управления ей. Среди них выделяются:
\begin{itemize}
    \item \textit{Amazon Web Services (AWS)} — один из крупнейших облачных провайдеров, предоставляющий широкий спектр услуг для развертывания и управления инфраструктурой.
    \item \textit{Yandex.Cloud} — облачная платформа, предлагающая облачные сервисы для создания и управления инфраструктурой с учетом специфики российского рынка.
    \item \textit{Azure} — облачная платформа от \textit{Microsoft}, предлагающая широкий набор сервисов для бизнеса.
    \item \textit{Google Cloud Platform (GCP)} — облачная платформа от компании \textit{Google}, которая предоставляет инструменты для развертывания и управления приложениями и инфраструктурой с высокой доступностью и масштабируемостью.
    \item \textit{IBM Cloud} — облачный провайдер, предлагающий решения для бизнеса, включая услуги по созданию и управлению контейнерами, виртуальными машинами, а также поддерживающий \textit{DevOps} инструменты.
    \item \textit{Alibaba Cloud} — облачная платформа от \textit{Alibaba Group}, предлагающая решения для хранения данных, вычислений и сети в облаке, с ориентированностью на азиатский рынок.
\end{itemize}

Каждый из этих облачных провайдеров имеет свои особенности и преимущества, в зависимости от требований бизнеса и региона. Например, \textit{AWS} является одним из самых мощных и универсальных провайдеров с глобальной инфраструктурой, в то время как \textit{Yandex.Cloud} может быть более предпочтительным для российских пользователей, из-за соответствия требованиям безопасности и локализации данных.

\subsubsection{\textit{Amazon Web Services}.}
\label{sec:aws}
\textit{Amazon Web Services (AWS)} — это ведущая облачная платформа, предоставляющая широкий спектр облачных услуг, включая вычислительные мощности, базы данных, аналитику, хранение данных, сетевые решения и многое другое. Она поддерживает как традиционные виртуальные машины, так и контейнеризированные приложения.

\subsubsection{Основные сервисы \textit{AWS}.}
\begin{itemize}
    \item \textit{EC2 (Elastic Compute Cloud)} — сервис для запуска виртуальных машин, предоставляющий масштабируемые вычислительные ресурсы.
    \item \textit{S3 (Simple Storage Service)} — объектное хранилище данных для хранения файлов и резервных копий.
    \item \textit{RDS (Relational Database Service)} — сервис для управления реляционными базами данных.
    \item \textit{EKS (Elastic Kubernetes Service)} — сервис для управления \textit{Kubernetes}-кластерами, позволяющий развертывать и управлять контейнеризированными приложениями.
    \item \textit{IAM (Identity and Access Management)} — сервис для управления пользователями и их доступом к ресурсам \textit{AWS}.
\end{itemize}

\subsubsection{Применение \textit{AWS}.}
\textit{AWS} используется для создания и управления масштабируемыми и высокодоступными приложениями, поддерживает DevOps-процессы и автоматизацию инфраструктуры. Он является одним из самых популярных облачных провайдеров для крупных организаций и стартапов, предоставляя гибкие решения для различных бизнес-задач.

\subsubsection{\textit{Yandex.Cloud}.}
\label{sec:yandex_cloud}
\textit{Yandex.Cloud} — это облачная платформа от российской компании \textit{Yandex}, которая предоставляет широкий набор сервисов для разработки, развертывания и управления приложениями и инфраструктурой в облаке.

\subsubsection{Основные сервисы Yandex.Cloud.}
\begin{itemize}
    \item \textit{Compute Cloud} — сервис для запуска виртуальных машин с возможностью масштабирования ресурсов.
    \item \textit{Managed Kubernetes} — управляемый сервис для развертывания и управления \textit{Kubernetes}-кластерами.
    \item \textit{Object Storage} — объектное хранилище для хранения данных, резервных копий и медиафайлов.
    \item \textit{Yandex.DB} — сервис для работы с реляционными и NoSQL базами данных.
    \item \textit{Identity and Access Management} (IAM) — система управления доступом пользователей и сервисов в облаке.
\end{itemize}

\subsubsection{Применение \textit{Yandex.Cloud}.}
\textit{Yandex.Cloud} активно используется компаниями, ориентированными на российский рынок и требующими соблюдения местных стандартов безопасности и законодательных требований. Платформа поддерживает все необходимые инструменты для разработки, развертывания и масштабирования приложений в облаке, включая возможности для работы с \textit{Kubernetes} и виртуальными машинами.
