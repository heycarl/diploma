\sectionCenteredToc{Введение}
\label{sec:intro}

Семимильными шагами ступает прогресс по планете Земля. Каждый день мы можем наблюдать, как в мире происходит так называемая <<цифровая революция>>, которая началась еще в последних десятилетиях прошлого века. Связана она с распространением информационных технологий и проникновением их во все сферы жизни общества.

Современные корпорации все больше зависят от надежных и высокодоступных информационных технологий для обеспечения бесперебойной работы своих бизнес-процессов. В условиях глобализации, быстрого роста объемов данных и возрастающих требований к безопасности, обеспечение отказоустойчивости и масштабируемости \textit{IT}-инфраструктуры становится ключевым фактором успеха для многих организаций. Компании требуют решений, которые могут обеспечить доступность сервисов и данных при любых обстоятельствах, даже в случае сбоя компонентов инфраструктуры.

Особое внимание стоит уделить разработке отказоустойчивых и масштабируемых систем, которые способны поддерживать высокие стандарты безопасности и обеспечивать удобный доступ для сотрудников. Системы, такие как \textit{Kubernetes}, \textit{Infrastructure as Code (IaC)}, \textit{Vault} для хранения секретов и \textit{GitOps} для автоматизации развертывания, становятся неотъемлемой частью корпоративной \textit{IT}-инфраструктуры, позволяя минимизировать риски и повышать оперативность реагирования на изменения в потребностях бизнеса.

Целью дипломного проекта является создание отказоустойчивой корпоративной \textit{IT}-инфраструктуры компании, которая обеспечит высокий уровень доступности сервисов, защиту данных и удобство для пользователей. В рамках проекта будет разработано решение, включающее в себя развертывание \textit{Kubernetes} кластера для оркестрации контейнеров, использование \textit{IaC} для управления облачной инфраструктурой, внедрение системы управления пользователями (\textit{IDM}) для централизованного контроля доступа и безопасную систему хранения секретов с использованием \textit{Vault}. Также будет реализована система \textit{VPN} для безопасного удаленного доступа и \textit{GitOps} для автоматизации развертывания приложений.

Кроме того, проект включает внедрение системы мониторинга и алертинга, что позволит своевременно выявлять и устранять потенциальные проблемы в инфраструктуре, минимизируя возможные простои. Важно отметить, что реализация таких компонентов в рамках данного проекта направлена на обеспечение высокой доступности, безопасности и надежности инфраструктуры компании, что позволит эффективно поддерживать ее работу в условиях растущих требований и нагрузок.

В процессе разработки будет выполнен анализ существующих решений и технологий для создания высокодоступных инфраструктур, изучены лучшие практики для обеспечения отказоустойчивости и безопасности данных. Особое внимание будет уделено проектированию архитектуры системы, которая должна быть гибкой, масштабируемой и легко адаптируемой к изменениям в потребностях бизнеса. Также будет проведено технико-экономическое обоснование выбора решений, с учетом их стоимости, эффективности и потенциальных рисков.

Задачи, поставленные в рамках дипломного проекта, позволят не только создать надежную и безопасную корпоративную \textit{IT}-инфраструктуру, но и значительно улучшить процессы управления и автоматизации в компании. Ожидается, что успешная реализация проекта обеспечит не только улучшение производительности и надежности, но и повысит степень удовлетворенности пользователей благодаря улучшенному доступу и безопасности.

Развитие технологий в последние десятилетия значительно повысило требования к корпоративным IT-системам. Растущий объем данных, увеличение числа пользователей и сервисов, а также необходимость обеспечивать непрерывность бизнес-процессов требуют внедрения решений, которые способны эффективно управлять масштабируемыми и высокодоступными инфраструктурами. В таких условиях отказоустойчивость и безопасность становятся основными приоритетами для любой компании, и создание таких решений требует применения передовых технологий и подходов.

Одним из ключевых факторов успешной реализации проекта является автоматизация процессов. С использованием \textit{GitOps}, \textit{IaC} и других современных инструментов возможно не только быстрое развертывание инфраструктуры, но и ее эффективное управление с минимальными затратами времени и ресурсов. Внедрение таких решений значительно сокращает время на развертывание новых приложений, улучшает управление конфигурациями и позволяет гибко реагировать на изменения в потребностях бизнеса. В этом контексте, создание отказоустойчивой и масштабируемой \textit{IT}-инфраструктуры способствует не только увеличению надежности системы, но и улучшению операционной эффективности компании в целом.