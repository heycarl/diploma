\setcounter{section}{6}
\section{Технико-экономическое обоснование разработки отказоустойчивой корпоративной \textit{IT}-инфраструктуры компании}

\subsection{Описание функций, назначения и потенциальных пользователей программного средства}

Разработанный в дипломном проекте сервис предназначен для создания отказоустойчивой корпоративной \textit{IT}-инфраструктуры компании, включающей в себя несколько ключевых компонентов, обеспечивающих высокую доступность, безопасность и удобство для сотрудников компании. Сервис поддерживает следующие основные функции:

\begin{itemize}
    \item система управления пользователями \textit{(IDM)}, позволяющая централизованно управлять доступом сотрудников к корпоративным ресурсам;
    \item корпоративный таск-трекер для эффективного планирования и отслеживания выполнения задач в рамках проектов;
    \item корпоративный менеджер паролей для безопасного хранения и управления паролями пользователей;
    \item система релизов программных продуктов, обеспечивающая автоматизацию процессов развертывания и обновления программных решений;
    \item \textit{VPN} для безопасного доступа сотрудников к корпоративной сети из любой точки мира;
    \item кластер баз данных, обеспечивающий отказоустойчивость и масштабируемость хранения данных;
    \item хранилище секретов \textit{(Vault)}, предназначенное для безопасного хранения и управления ключами, сертификатами и другими чувствительными данными;
    \item кластер \textit{Kubernetes (k8s)}, обеспечивающий оркестрацию контейнеризированных приложений и их высокую доступность.
\end{itemize}

С точки зрения экономической эффективности для разработчика, прибыль заключается в том, что данное приложение является индивидуальным заказом для корпоративного клиента, что позволяет оптимизировать процессы в компании, снизить риски сбоев в инфраструктуре и повысить безопасность всех бизнес-процессов.

Программное средство позволит заказчику использовать его функциональность для решения следующих задач:
\begin{itemize}
    \item централизованное управление доступом и безопасностью пользователей в корпоративной среде;
    \item автоматизация процессов развертывания и обновления программного обеспечения, что значительно ускоряет релизы и повышает их качество;
    \item обеспечение безопасного хранения и управления паролями и ключами, улучшая защиту от утечек данных;
    \item создание гибкой и масштабируемой \textit{IT}-инфраструктуры с использованием современных технологий, таких как \textit{Kubernetes} и базы данных с высокой доступностью.
\end{itemize}

Данный программный модуль ориентирован на IT-отделы крупных организаций и компании, что предполагает большой охват пользователей, включая системных администраторов, инженеров \textit{DevOps}, разработчиков программного обеспечения и специалистов по безопасности. Высокая потребность в данном программном решении обусловлена необходимостью создания стабильной и безопасной корпоративной инфраструктуры, обеспечивающей бесперебойную работу всех сервисов и минимизацию рисков сбоев, что, в свою очередь, способствует повышению производительности и снижению затрат на поддержку инфраструктуры. Тенденции в области информационных технологий все больше ориентируются на построение отказоустойчивых и безопасных систем, что делает создание таких решений приоритетом для современных организаций.


\subsection{Расчет затрат на разработку программного средства}
\newcommand{\sreengineer}{Инженер доступности}
\newcommand{\programmer}{Инженер-программист}
\newcommand{\businessanalytic}{Бизнес-аналитик}

\subsubsection{Расчет затрат на основную заработную плату разработчиков.}
Для расчета заработной платы используем формулу \eqref{sec_economy:eq:main_salary}:
\begin{equation}
    \label{sec_economy:eq:main_salary}
    \text{З}_{\text{о}} = \text{K}_{\text{пр}}\sum_{i=1}^n\text{З}_{\text{ч.}i} \times t_i \text{\,,}
\end{equation}
\begin{explanationx}
    \item[где] $ n $ -- количество исполнителей, занятых разработкой ПО;
    \item $ \text{K}_{\text{пр}} $ -- коэффициент, учитывающий процент премий;
    \item $ \text{З}_{\text{ч.}i} $ -- часовая заработная плата i-го исполнителя, р.;
    \item $ t_i $ -- трудоемкость работ, выполняемых i-м исполнителем, ч.
\end{explanationx}

Размер месячной заработной платы исполнителя каждой категории соответствует сложившемуся на рынке труда размеру заработной платы для категорий работников, участвующих в разработке~\cite{programmersalaries}.
Кроме того, следует учитывать, что в ходе разработки проекта возможно изменение базовых показателей, таких как ежемесячная заработная плата или уровень премии. Эти изменения могут зависеть от множества факторов, таких как изменения в законодательстве о труде или внутренние корпоративные политики.
Эти данные были собраны с учетом актуальных рыночных условий и отраслевых стандартов. Для расчета часовой заработной платы использовалось стандартное количество рабочих часов в месяце -— 168, что соответствует 40-часовой рабочей неделе. Размер премии был выбран равным 20\% от размера заработной платы. Такая премия является стандартной для большинства отраслей и служит дополнительным стимулом для повышения эффективности работы и выполнения задач в срок. Расчет затрат на основную заработную плату представлен в таблице \ref{sec_economy:table:main_salary}.

\FPeval{\workHoursPerMonth}{168}
\FPeval{\workDaysPerMonth}{21}

\FPeval{\sreSalary}{8300}
\FPeval{\sreWorkDays}{20}
\FPeval{\sreWorkHours}{40}
\FPeval{\sreSalaryByDay}{trunc(\sreSalary / \workDaysPerMonth, 2)}
\FPeval{\sreMainSalary}{trunc(\sreSalaryByDay * \sreWorkDays, 2)}

\FPeval{\programmerSalary}{4000}
\FPeval{\programmerSalaryByDay}{trunc(\programmerSalary / \workDaysPerMonth, 2)}
\FPeval{\programmerWorkDays}{20}
\FPeval{\programmerMainSalary}{trunc(\programmerSalaryByDay * \programmerWorkDays, 2)}

\FPeval{\baSalary}{2500}
\FPeval{\baSalaryByDay}{trunc(\baSalary / \workDaysPerMonth, 2)}
\FPeval{\baWorkDays}{5}
\FPeval{\baMainSalary}{trunc(\baSalaryByDay * \baWorkDays, 2)}

\FPeval{\commonMainSalary}{trunc(\sreMainSalary + \programmerMainSalary + \baMainSalary, 2)}
\FPeval{\bonusPercents}{20}
\FPeval{\bonusSize}{trunc(\commonMainSalary * \bonusPercents / 100, 2)}
\FPeval{\commonMainSalaryWithBonus}{trunc(\commonMainSalary + \bonusSize, 2)}
\begin{longtable}{
    | >{\raggedright\arraybackslash}m{0.350\textwidth}
    | >{\centering\arraybackslash}m{0.13\textwidth}
    | >{\centering\arraybackslash}m{0.14\textwidth}
    | >{\centering\arraybackslash}m{0.12\textwidth}
    | >{\centering\arraybackslash}m{0.13\textwidth}|}

    \caption{Затраты на основную заработную плату разработчиков}
    \label{sec_economy:table:main_salary} \\
    \hline
    \centering\arraybackslash Наименование категории работника и должности &
    \centering\arraybackslash Месячная за\-ра\-бот\-ная плата, р. &
    \centering\arraybackslash За\-ра\-бот\-ная плата за день, р. &
    \centering\arraybackslash Тру\-доём\-кость работ, дни. &
    \centering\arraybackslash Сумма, р. \\
    \hline
    \endfirsthead

    \continueTableCaption \\
    \hline
    \centering\arraybackslash Наименование категории работника и должности &
    \centering\arraybackslash Месячная за\-ра\-бот\-ная плата, р. &
    \centering\arraybackslash За\-ра\-бот\-ная плата за день, р. &
    \centering\arraybackslash Тру\-доём\-кость работ, дни. &
    \centering\arraybackslash Сумма, р. \\
    \hline
    \endhead

    \hline
    1. \sreengineer &
    \sreSalary &
    \sreSalaryByDay &
    \sreWorkDays &
    \sreMainSalary
    \\
    
    \hline
    2. \programmer &
    \programmerSalary &
    \programmerSalaryByDay &
    \programmerWorkDays &
    \programmerMainSalary
    \\

    \hline
    3. \businessanalytic &
    \baSalary &
    \baSalaryByDay &
    \baWorkDays &
    \baMainSalary
    \\

    \hline
    \multicolumn{4}{|l|}{Итого} &
    \commonMainSalary
    \\

    \hline
    \multicolumn{4}{|l|}{Премия (\bonusPercents)\%} &
    \bonusSize
    \\

    \hline
    \multicolumn{4}{|l|}{Всего основная заработная плата} &
    \commonMainSalaryWithBonus
    \\
    \hline
\end{longtable}


\subsubsection{Расчет дополнительных выплат, предусмотренных законодательством о труде, осуществляем по формуле \eqref{sec_economy:eq:additional_salary}:.}
\begin{equation}
    \label{sec_economy:eq:additional_salary}
    \text{З}_{\text{д}} = \frac{\text{З}_{\text{о}} \times \text{Н}_{\text{д}}}{100} \text{\,,}
\end{equation}
\begin{explanationx}
    \item[где] $ \text{З}_{\text{о}} $ -- затраты на основную заработную плату, р.;
    \item $ \text{Н}_{\text{д}} $ -- норматив дополнительной заработной платы.
\end{explanationx}

\FPeval{\NormAdditionalSalary}{15}
\FPeval{\AdditionalSalary}{trunc(\commonMainSalaryWithBonus * \NormAdditionalSalary / 100, 2)}
Норматив дополнительной заработной платы был принят равным \NormAdditionalSalary\%.
Размер дополнительной заработной составил:\\
\begin{equation*}
    \text{З}_{\text{д}} = \frac{\text{З}_{\text{о}} \times \text{Н}_{\text{д}}}{100} = \frac{\commonMainSalaryWithBonus \times \NormAdditionalSalary}{100} = \AdditionalSalary \text{ р.}
\end{equation*}

\subsubsection{Расчет отчислений на социальные нужды.}

Расчет производим в соответствии с действующими законодательными
актами по формуле \eqref{sec_economy:eq:social_payments}:
\begin{equation}
    \label{sec_economy:eq:social_payments}
    \text{Р}_{\text{соц}} = \frac{ (\text{З}_{\text{о}} + \text{З}_{\text{д}}) \times \text{Н}_{\text{соц}} }{100} \text{\,,}
\end{equation}
\begin{explanationx}
    \item[где] $ \text{Н}_{\text{соц}} $ -- норматив отчислений от фонда оплаты труда.
\end{explanationx}

\FPeval{\socialPaymentsPercent}{35}
\FPeval{\socialPayments}{trunc((\commonMainSalaryWithBonus + \AdditionalSalary) * \socialPaymentsPercent / 100, 2)}
Норматив отчислений от фонда оплаты труда был принят равным \socialPaymentsPercent\%. Размер отчислений на социальные нужны составил:\\
\begin{equation*}
    \text{Р}_{\text{соц}} = \frac{ (\text{З}_{\text{о}} + \text{З}_{\text{д}}) \times \text{Н}_{\text{соц}} }{100} = \frac{ (\commonMainSalaryWithBonus + \AdditionalSalary) \times \socialPaymentsPercent }{100} = \socialPayments \text{ р.}
\end{equation*}

\subsubsection{Расчет прочих затрат.}

Прочие затраты расчитываем по формуле \eqref{sec_economy:eq:other_payments}:
\begin{equation}
    \label{sec_economy:eq:other_payments}
    \text{Р}_\text{пз} = \frac{\text{З}_\text{о} \times \text{Н}_\text{пз}}{100} \text{\,,}
\end{equation}
\begin{explanationx}
    \item[где] $ \text{Н}_\text{пз} $ -- норматив прочих затрат.
\end{explanationx}

\FPeval{\otherPaymentsNorm}{100}
\FPeval{\otherPaymentsTotal}{trunc(\commonMainSalaryWithBonus * \otherPaymentsNorm / 100, 2)}
Норматив прочих затрат был принят равным 100\%. Размер прочих затрат составил:\\
\begin{equation*}
    \label{sec_economy:eq:other_payments}
    \text{Р}_\text{пз} = \frac{\text{З}_\text{о} \times \text{Н}_\text{пз}}{100} = \frac{\commonMainSalaryWithBonus \times \otherPaymentsNorm}{100} = \otherPaymentsTotal \text{ р.}
\end{equation*}

\subsubsection{Расчет общих затрат на разработку.}
\FPeval{\totalPayments}{trunc(\commonMainSalaryWithBonus + \AdditionalSalary + \socialPayments + \otherPaymentsTotal, 2)}

Расчет общих затрат на разработку расчитывается по формуле \eqref{sec_economy:eq:total_payments}:
\begin{equation}
    \label{sec_economy:eq:total_payments}
    \text{З}_\text{р} = \text{З}_\text{о} + \text{З}_\text{д} + \text{Р}_\text{соц} + \text{Р}_\text{пз}\text{\,,}
\end{equation}
После вычисления получаем:\\
\begin{equation*}
    \begin{gathered}
    \text{З}_\text{р} = \text{З}_\text{о} + \text{З}_\text{д} + \text{Р}_\text{соц} + \text{Р}_\text{пз} = \\
    = \commonMainSalaryWithBonus + \AdditionalSalary + \socialPayments + \otherPaymentsTotal = \totalPayments
    \end{gathered}
\end{equation*}

\subsubsection{Расчет плановой прибыли, включаемой в цену продукта.}

Для расчета плановой прибыли, включаемой в цену программного модуля, требуется воспользоваться формулой \eqref{sec_economy:eq:product_cost}:
\begin{equation}
    \label{sec_economy:eq:product_cost}
    \text{П}_\text{п.с.} = \frac{\text{З}_\text{р} \times \text{Р}_\text{п.с.}}{100}\text{\,,}
\end{equation}
\begin{explanationx}
    \item[где] $ \text{З}_\text{р} $ -- затраты на разработку программного модуля, р.;
    \item $ \text{Р}_\text{п.с.} $ -- рентабельность затрат на разработку программного средства.
\end{explanationx}
\FPeval{\rentAbility}{30}
\FPeval{\rentPlan}{trunc(\totalPayments * \rentAbility / 100, 2)}

Рентабельность затрат на разработку принята \rentAbility\%.\\
\begin{equation*}
    \label{sec_economy:eq:product_cost_calc}
    \text{П}_\text{п.с.} = \frac{\text{З}_\text{р} \times \text{Р}_\text{п.с.}}{100} = \frac{\totalPayments \times \rentAbility}{100} = \rentPlan\text{ р.}
\end{equation*}

\subsubsection{Расчет отпускной цены программного средства.} Для расчета отпускной цены программного средства стоит воспользоваться формулой \eqref{sec_economy:eq:release_cost}:
\begin{equation}
    \label{sec_economy:eq:release_cost}
    \text{Ц}_\text{п.с.} = \text{З}_\text{р} + \text{П}_\text{п.с.}\text{\,,}
\end{equation}

\FPeval{\releaseCost}{trunc(\totalPayments + \rentPlan, 2)}
После вычислений получаем отпускную цену товара.\\
\begin{equation*}
    \text{Ц}_\text{п.с.} = \text{З}_\text{р} + \text{П}_\text{п.с.} = \totalPayments + \rentPlan = \releaseCost\text{ р.}
\end{equation*}


В результате вычисления всех затрат и рассчитав все величины сформируем итоговую таблицу затрат на разработку программного модуля. Резальутаты приведены в таблице \ref{sec_economy:table:payments}.

\begin{longtable}{ % 0.93
    | >{\raggedright\arraybackslash}m{0.21\textwidth}
    | >{\centering\arraybackslash}m{0.6\textwidth}
    | >{\centering\arraybackslash}m{0.12\textwidth}|}

    \caption{Затраты на разработку программного модуля}
    \label{sec_economy:table:payments} \\
    \hline
    \centering\arraybackslash Наименование статьи затрат &
    \centering\arraybackslash Формула/таблица для расчета &
    \centering\arraybackslash Значение, р. \\
    \hline
    \endfirsthead

    \continueTableCaption \\
    \hline
    \centering\arraybackslash Наименование статьи затрат &
    \centering\arraybackslash Формула/таблица для расчета &
    \centering\arraybackslash Значение, р. \\
    \hline
    \endhead

    Основная заработная плата разработчиков &
    смотреть таблицу \ref{sec_economy:table:main_salary} &
    \commonMainSalaryWithBonus
    \\

    \hline
    Дополнительная заработная плата разработчиков &
    $ \text{З}_{\text{д}} = \frac{\commonMainSalaryWithBonus \times \NormAdditionalSalary}{100} $ &
    \AdditionalSalary
    \\

    \hline
    Отчисления на социальные нужды  &
    $ \text{Р}_{\text{соц}} = \frac{ (\commonMainSalaryWithBonus + \AdditionalSalary) \times \socialPaymentsPercent }{100} $ &
    \socialPayments
    \\

    \hline
    Прочие расходы &
    $ \text{Р}_\text{пз} = \frac{\commonMainSalaryWithBonus \times \otherPaymentsNorm}{100} $ &
    \otherPaymentsTotal
    \\

    \hline
    Общая сумма затрат на разработку &
    $ \text{З}_\text{р} = \commonMainSalaryWithBonus + \AdditionalSalary + \socialPayments + \otherPaymentsTotal $ &
    \totalPayments
    \\

    \hline
    Плановая прибыль, включаемая в цену программного модуля  &
    $ \text{П}_\text{п.с.} = \frac{\totalPayments \times \rentAbility}{100} $ &
    \rentPlan
    \\

    \hline
    Отпускная цена программного средства &
    $ \text{Ц}_\text{п.с.} = \totalPayments + \rentPlan $ &
    \releaseCost
    \\
    \hline
\end{longtable}

\subsection{Расчет результата от разработки и использования программного средства}

Экономический эффект от разработки и использования программного средства рассчитывается как для организации-разработчика, так и для организации-заказчика.

\subsubsection{Расчет результата для организации-разработчика.}
Поскольку программное средство будет реализовываться по отпускной цене,
то экономический эффект для организации-разработчика определяется как прирост
чистой прибыли по формуле \eqref{sec_economy:eq:profit_development}:
\begin{equation}
    \label{sec_economy:eq:profit_development}
    \Delta \text{П}_\text{ч} = \text{П}_\text{п.с.} \cdot \biggl( 1 -
    \frac{\text{Н}_\text{п}}{100} \biggr),
\end{equation}
\begin{explanationx}
    \item[где] $ \text{Н}_\text{п} $ -- ставка налога на прибыль;
    \item $ \text{П}_\text{п.с.} $ -- прибыль, включаемая в цену программного средства, р.
\end{explanationx}

\FPeval{\valNPPercent}{20}

Организация-разработчик не является резидентом Парка высоких технологий, поэтому обязана уплатить налог на прибыль, равный $ \valNPPercent\% $ согласно Налоговому кодексу Республики Беларусь по состоянию на март 2025 года.\\
\FPeval{\econEffect}{trunc(\rentPlan * (1 - \valNPPercent/100), 2)}
\begin{equation*}
    \Delta \text{П}_\text{ч} = \rentPlan \cdot \biggl( 1 -
    \frac{\valNPPercent}{100} \biggr) = \econEffect \text{ р.}
\end{equation*}

\subsubsection{Расчет результата для организации-заказчика.}

Разрабатываемое программное средство позволяет сэкономить на заработной плате
и начислениях на заработную плату сотрудников за счет снижения трудоемкости работ.

Для расчета экономии на заработной плате воспользуемся формулой \eqref{sec_economy:eq:salary_economy}:

\begin{equation}
  \label{sec_economy:eq:salary_economy}
  \text{Э}_\text{з.п} = \text{К}_\text{пр} \cdot
    \bigl(t_\text{р}^\text{без п.с} - t_\text{р}^\text{с п.с} \bigr) \cdot
    \text{Т}_\text{ч} \cdot N_\text{п} \cdot
    \biggl( 1 + \frac{\text{Н}_\text{д}}{100} \biggr) \cdot
    \biggl( 1 + \frac{\text{Н}_\text{соц}}{100} \biggr),
\end{equation}
\begin{explanationx}
  \item[где] $ \text{К}_\text{пр} $ -- коэффициент премий;
  \item $ t_\text{р}^\text{без п.с} $ -- трудоемкость выполнения работ сотрудниками до внедрения программного средства, ч;
  \item $ t_\text{р}^\text{с п.с} $ -- трудоемкость выполнения работ сотрудниками после внедрения программного средства, ч;
  \item $ \text{T}_\text{ч} $ -- часовой оклад (часовая тарифная ставка) сотрудника,
  использующего программное средство, р.;
  \item $ N_\text{п} $ -- плановый объем работ, выполняемых сотрудником.
\end{explanationx}

\FPeval{\valTargetKPr}{1.75}
\FPeval{\valTargetTCh}{16.36}
\FPeval{\valTargetNP}{1}
\FPeval{\valTargetTBefore}{3500}
\FPeval{\valTargetTAfter}{2200}
\FPeval{\valEZP}{trunc(\valTargetKPr * (\valTargetTBefore - \valTargetTAfter) *
  \valTargetTCh * \valTargetNP * (1 + \NormAdditionalSalary / 100) *
  (1 + \socialPaymentsPercent / 100), 2)}

Определим коэффициент премий у организации-заказчика равным $ \text{К}_\text{пр} = \valTargetKPr $.

Часовой оклад будем считать равным $ \text{T}_\text{ч} = \valTargetTCh \text{ р.} $

Плановый объем работ, выполняемых сотрудником, примем равным $ N_\text{п} = \valTargetNP $.

Трудоемкость выполнения работ сотрудниками до внедрения примем равным $ t_\text{р}^\text{без п.с} = \valTargetTBefore \text{ ч} $, а после -- $ t_\text{р}^\text{с п.с} = \valTargetTAfter \text{ ч} $.

Используя найденные значения, определим экономию для организации-заказчика по
формуле \eqref{sec_economy:eq:salary_economy}:\\
\begin{equation*}
  \label{sec_economy:eq:calc}
  \begin{split}
  % Конец формулы будет на &
  \text{Э}_\text{з.п} = \valTargetKPr \cdot
    \bigl(\valTargetTBefore - \valTargetTAfter \bigr) \cdot
    \valTargetTCh \cdot \valTargetNP \cdot
    \biggl( & 1 + \frac{\NormAdditionalSalary}{100} \biggr) \times \\
    \times
    \biggl( 1 + \frac{\socialPaymentsPercent}{100} \biggr) =
    \valEZP \text{ р.}
  \end{split}
\end{equation*}

\FPeval{\valETek}{\valEZP}
Источником экономии на текущих затратах является экономия на заработной плате сотрудников. Поэтому
$ \text{Э}_\text{тек} = \text{Э}_\text{з.п} = \valETek \text{ р.} $

Экономическим эффектом является прирост чистой прибыли, полученной за счет экономии на текущих затратах предприятия, которое определяется по формуле:
\begin{equation}
    \label{sec_economy:eq:target_delta_pch}
    \Delta \text{П}_\text{ч} = \bigl(\text{Э}_\text{тек} -
    \Delta \text{З}_\text{тек}^\text{п.с} \bigr)
    \cdot \biggl( 1 - \frac{\text{Н}_\text{п}}{100} \biggr),
\end{equation}
\begin{explanationx}
  \item[где] $ \text{Э}_\text{тек} $ -- экономия на текущих затратах при использовании программного средства, р.;
  \item $ \Delta \text{З}_\text{тек}^\text{п.с} $ -- прирост текущих затрат, связанных с использованием программного средства, р.
\end{explanationx}

\FPeval{\valDeltaZTek}{0}
\FPeval{\valTargetDeltaPCh}{round((\valETek - \valDeltaZTek) *
  (1 - \valNPPercent / 100), 2)}

Поскольку установка программного средства будет происходить на новых устройствах,
то никаких дополнительных расходов по обновлению и установке не будет. Следовательно,
$ \Delta \text{З}_\text{тек}^\text{п.с} = \valDeltaZTek \text{ р.} $

Используя определенные значения, найдем прирост чистой прибыли для
организации-заказчика по формуле \eqref{sec_economy:eq:target_delta_pch}:\\
\begin{equation*}
    \label{sec_economy:eq:target_delta_pch_calc}
    \Delta \text{П}_\text{ч} = ( \valETek - \valDeltaZTek )
    \biggl( 1 - \frac{\valNPPercent}{100} \biggr) =
    \valTargetDeltaPCh \text{ р.}
\end{equation*}

\subsection{Расчет показателей экономической эффективности разработки и использования программного средства}

\subsubsection{Расчет показателей экономической эффективности для организации-разработчика.}

При вычислении эффективности разработки воспользуемся формулой \eqref{sec_economy:eq:effectivity}
\begin{equation}
    \label{sec_economy:eq:effectivity}
    \text{Р}_\text{и} = \frac{\Delta\text{П}_\text{ч}}{\text{З}_\text{р}} \times 100\text{\%\,,}
\end{equation}
\begin{explanationx}
    \item[где] $ \Delta\text{П}_\text{ч} $ -- прирост чистой прибыли, полученной от разработки программного обеспечения организацией разработчиком, по индивидуальному заказу, р.;
    \item $ \text{З}_\text{р} $ -- затраты на разработку программного средства организацией-разработчиком, р.
\end{explanationx}
\FPeval{\effectivityDev}{trunc(\valTargetDeltaPCh / \otherPaymentsTotal * 100, 2)}
В результате получаем:\\
\begin{equation*}
    \text{Р}_\text{и} = \frac{\Delta\text{П}_\text{ч}}{\text{З}_\text{р}} \times 100\text{\%} = \\
    = \frac{\valTargetDeltaPCh}{\otherPaymentsTotal} \times 100\text{\%} = \effectivityDev \text{\%}
\end{equation*}

\subsubsection{Расчет показателей экономической эффективности для организации-заказчика.}

Поскольку сумма инвестиций меньше суммы годового экономического эффекта,
то оценка экономической эффективности инвестиций в разработку программного средства
осуществляется с помощью расчета нормы прибыли по формуле:
\newline
\begin{equation}
    \label{sec_economy:eq:target_ri}
    \text{Р}_\text{и} = \frac{\Delta \text{П}_\text{ч}}{\text{Ц}_\text{пс}
        \cdot \bigl( 1 + \frac{\text{Н}_\text{д.с}}{100} \bigr) }
        \cdot 100 \ \text{\%},
\end{equation}
\begin{explanationx}
  \item[где] $ \text{Н}_\text{д.с} $ -- ставка налога на добавленную стоимость.
\end{explanationx}

\FPeval{\valNdsPercent}{20}
\FPeval{\valTargetRi}{round(\valTargetDeltaPCh / (\releaseCost * (1 + \valNdsPercent
  / 100)) * 100, 2)}

По состоянию на март 2025 года, ставка налога на добавленную стоимость составляет $ \valNdsPercent \ \% $.

Используя ранее найденные значения прироста чистой прибыли
и цену программного средства, определим значение нормы прибыли
по формуле \eqref{sec_economy:eq:target_ri}:\\
\begin{equation*}
    \label{sec_economy:eq:target_ri_calc}
    \text{Р}_\text{и} = \frac{\valTargetDeltaPCh}{\releaseCost
        \cdot \bigl( 1 + \frac{\valNdsPercent}{100} \bigr) }
        \cdot 100 = \valTargetRi \ \text{\%}.
\end{equation*}

\subsection{Вывод об экономической эффективности}

В результате технико-экономического обоснования были выполнены расчеты инвестиций в разработку программного средства. Общая сумма затрат на разработку программного средства составила $ \totalPayments $ рублей, отпускная цена $ \releaseCost $ рублей, чистая прибыль организации разработчика $ \rentPlan $ рублей, простая норма прибыли организации разработчика $ \effectivityDev $ \% и простая норма прибыли организации заказчика $ \valTargetRi $\%.
Можно сказать, что разработка и внедрении данного программного средства подтверждает эффективность. Инвестиции, вложенные в его разработку, окупятся за счет уменьшения трудоемкости работ, что приведет к снижению затрат на оплату труда сотрудников.
Для организации-разработчика данная разработка является экономически оправданной, так как полученная прибыль соответствует средним показателям рентабельности. Также данное программное средство выгодно получить от организации-разработчика, так как организация-заказчик не обладает необходимыми кадрами для разработки подобного программного средства.