\referenceTitle
Дипломный проект представлен следующим образом. Электронные носители: 1 компакт-диск. Чертежный материал: 6 листов формата А1. Пояснительная записка: 124 страниц, 25 рисунков, 2 таблицы, 15 литературных источников, 3 приложения.

Ключевые слова: отказоустойчивая \textit{IT}-инфраструктура, корпоративная сеть, мониторинг, управление инцидентами, кластер \textit{Kubernetes}, автоматизация, безопасность, высокодоступные системы, логирование, метрики.

Предметной областью данного проекта является построение отказоустойчивой корпоративной \textit{IT}-инфраструктуры.

Объектом разработки является комплексное программно-аппаратное решение, обеспечивающее устойчивую и безопасную работу корпоративных сервисов при сбоях и нагрузках.

Целью данного дипломного проекта является разработка и внедрение отказоустойчивой \textit{IT}-инфраструктуры для повышения надежности и безопасности корпоративных информационных систем.

Для разработки использовались технологии \textit{Kubernetes} для оркестрации контейнеров, система мониторинга \textit{VictoriaMetrics}, стек логирования \textit{Loki} + \textit{Promtail}, инструменты управления секретами \textit{HashiCorp Vault}, а также средства автоматизации и управления конфигурациями.

Результатом разработки является полнофункциональная \textit{IT}-инфраструктура, обеспечивающая высокую доступность сервисов, централизованный сбор и анализ метрик и логов, эффективное управление доступом и автоматическое оповещение о инцидентах.

Разработанное решение имеет широкое применение в корпоративных системах, позволяя значительно снизить риски простоев и повысить уровень безопасности.

Проект является экономически эффективным и способствует оптимизации процессов эксплуатации \textit{IT}-инфраструктуры.

Дипломный проект полностью реализует все поставленные задачи проектирования и обеспечивает необходимый функционал. Возможна дальнейшая доработка для расширения мониторинговых метрик, улучшения автоматизации реагирования на инциденты и масштабирования инфраструктуры.

\newpage